\documentclass[twoside,openright,titlepage,fleqn,
	headinclude,11pt,a4paper,BCOR5mm,footinclude
	]{scrbook}
%--------------------------------------------------------------
        \newcommand{\myTitle}{Some Lisp code written studying
        masterpiece books\xspace}
% use the right myDegree option
\newcommand{\myDegree}{Self Studies\xspace}
%\newcommand{\myDegree}{
	%Corso di Laurea Specialistica in Scienze e Tecnologie 
	%dell'Informazione\xspace}
\newcommand{\myName}{Massimo Nocentini\xspace}
\newcommand{\myProf}{Pierluigi Crescenzi\xspace}
\newcommand{\myOtherProf}{Nome Cognome\xspace}
\newcommand{\mySupervisor}{Nome Cognome\xspace}
\newcommand{\myFaculty}{
	Facolt\`a di Scienze Matematiche, Fisiche e Naturali\xspace}
\newcommand{\myDepartment}{
	Dipartimento di Sistemi e Informatica\xspace}
\newcommand{\myUni}{\protect{
	Universit\`a degli Studi di Firenze}\xspace}
\newcommand{\myLocation}{Firenze\xspace}
\newcommand{\myTime}{2011-2012\xspace}
\newcommand{\myVersion}{Version 0.1\xspace}

%--------------------------------------------------------------
\usepackage[latin1]{inputenc} 
\usepackage[T1]{fontenc} 
\usepackage[square,numbers]{natbib} 
\usepackage[fleqn]{amsmath}  
\usepackage[english]{babel}
%--------------------------------------------------------------
\usepackage{dia-classicthesis-ldpkg} 
%--------------------------------------------------------------
% Options for classicthesis.sty:
% tocaligned eulerchapternumbers drafting linedheaders 
% listsseparated subfig nochapters beramono eulermath parts 
% minionpro pdfspacing
\usepackage[eulerchapternumbers,subfig,beramono,eulermath,
	parts]{classicthesis}
%--------------------------------------------------------------
\newlength{\abcd} % for ab..z string length calculation
% how all the floats will be aligned
\newcommand{\myfloatalign}{\centering} 
\setlength{\extrarowheight}{3pt} % increase table row height
\captionsetup{format=hang,font=small}
%--------------------------------------------------------------
% Layout setting
%--------------------------------------------------------------
\usepackage{geometry}
\geometry{
	a4paper,
	ignoremp,
	bindingoffset = 1cm, 
	textwidth     = 13.5cm,
	textheight    = 21.5cm,
	lmargin       = 3.5cm, % left margin
	tmargin       = 4cm    % top margin 
}
%--------------------------------------------------------------
\usepackage{listings}
\usepackage{hyperref}
% My Theorem
\newtheorem{oss}{Observation}[section]
\newtheorem{exercise}{Exercise}[section]
\newtheorem{thm}{Theorem}[section]
\newtheorem{cor}[thm]{Corollary}

\newtheorem{lem}[thm]{Lemma}

\newcommand{\vect}[1]{\boldsymbol{#1}}

% questo comando e' relativo alle correzioni che puo
% apportare il prof se lo desidera.
\newcommand{\prof}[1]{\boldsymbol{#1}}

% instead of boldsymbol I can use the arrow above the letter with
%\newcommand{\vect}[1]{\vec{#1}}

% page settings
% \pagestyle{headings}
%--------------------------------------------------------------
\begin{document}
\frenchspacing
\raggedbottom
\pagenumbering{roman}
\pagestyle{plain}
%--------------------------------------------------------------
% Frontmatter
%--------------------------------------------------------------
%--------------------------------------------------------------
% titlepage.tex (use thesis.tex as main file)
%--------------------------------------------------------------
\begin{titlepage}
	\begin{center}
   	\large
      \hfill
      \vfill
      \begingroup
			\spacedallcaps{\myUni} \\ 
			\myFaculty \\
			\myDegree \\ 
			\vspace{0.5cm}
         \includegraphics[scale=.065]{logo/unifi}\\
         \vspace{0.5cm}    
         Lab experiments    
      \endgroup 
      \vfill 
      \begingroup
      	\color{Maroon}\spacedallcaps{\myTitle} \\ \bigskip
      \endgroup
      \spacedlowsmallcaps{\myName}
      \vfill  
      %Relatore: \itshape{\myProf}
      \vfill                   
      \myTime
      \vfill                      
	\end{center}        
\end{titlepage}   
%--------------------------------------------------------------
% back titlepage
%--------------------------------------------------------------
   \newpage
	\thispagestyle{empty}
	\hfill
	\vfill
	\noindent\myName: 
	\textit{\myTitle,} 
	\myDegree, \textcopyright\ \myTime
%--------------------------------------------------------------
% back titlepage end
%--------------------------------------------------------------
\pagestyle{scrheadings}
%--------------------------------------------------------------
% Mainmatter
%--------------------------------------------------------------
\pagenumbering{arabic}

% settings for the lstlisting environment
\lstset{
	language = Lisp
	, numbers = right
	, basicstyle=\footnotesize
	, frame=tlbr
	, tabsize=2
	, captionpos=b
	, breaklines=true
	, showspaces=false
	, showstringspaces=false
}

\tableofcontents

\newpage

\chapter{The Little Schemer}
\label{chap:the-little-schemer}

\section{Lisp Functions' definitions}
\label{sec:functions-definitions}
\lstinputlisting{the-little-schemer/the-little-schemer.lisp}

% \section{Questions as assertions}
% \label{sec:questions-as-assertions}
% \lstinputlisting{the-little-schemer/questions-answers-as-unittests.lisp}

\section{Java implementation of Y-combinator}
\label{sec:java-implementation-of-y-combinator}
This implementation may seem repetitive and not object oriented: I've
choosen to do it in this way because every step is immediately
available without resorting to Git history.

\lstinputlisting[language=java]{the-little-schemer/y-combinator-java-implementation/src/ListLengthCalculatorHighOrder.java}
\lstinputlisting[language=java]{the-little-schemer/y-combinator-java-implementation/src/ListLengthCalculatorMaker.java}
\lstinputlisting[language=java]{the-little-schemer/y-combinator-java-implementation/src/ListLengthCalculatorRecursiveInvocation.java}
\lstinputlisting[language=java]{the-little-schemer/y-combinator-java-implementation/src/ListLengthCalculator.java}
\lstinputlisting[language=java]{the-little-schemer/y-combinator-java-implementation/src/ListLengthCalculators.java}
\lstinputlisting[language=java]{the-little-schemer/y-combinator-java-implementation/src/ListModule.java}

\lstinputlisting[language=java]{the-little-schemer/y-combinator-java-implementation/src/Ycombinator/Ycombinator.java}
\lstinputlisting[language=java]{the-little-schemer/y-combinator-java-implementation/src/Unittests.java}

%\Input{bibliography.tex}

\end{document} 

%%% Local Variables: 
%%% mode: latex
%%% TeX-master: t
%%% End: 
